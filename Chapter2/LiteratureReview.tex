% Chapter 2 - This tex file refers to the content of your second chapter  
%=======================================================================================

\section{Urban sensing applications}
Numerous novel urban sensing applications have been developed
recently. For example, Lian et al. \cite{p4} studied
the problem of restaurant survival prediction by considering geographical 
information and user mobility. They explored various factors under the guidance of the 
following considerations: (H1) the geographical placement of the
store play an important role in the store’s operation; (H2)
people’s offline mobility patterns to the store as well as its nearby places influence the business;
 (H3) user’s rating scores
(e.g., on Yelp) are explicit evaluations of the store from the
customers’ point of view; (H4) besides well-formatted rating
scores, review words contain more rich information which a
simple numeric score does not cover.

\noindent Wang et al. \cite{p5} proposed to spot and trace the latent trip purposes of taxi
trajectories from a city. They identified a very important
property of human mobility, which is, human mobility synchronization. In other words, 
if two regions share similar spatial configurations and urban functions in a particular 
time period, the two regions are likely to have similar patterns of arrival events.

\noindent Wu et al.\cite{p6} proposed an end-to-end Deep Event Attendance
Prediction (DEAP) framework—a three-level hierarchical LSTM architecture—to explicitly model 
users’ multi-dimensional and evolving preferences. DEAP explores the rich contextual 
information of events to address the aforementioned event cold-start challenge. In specific, its 
first level transforms events’ contextual information into latent embedding vectors in a non-linear way. 
In the second level, they aimed to encode the evolving exclusive preferences of users by considering their
 attendance behavior across different groups. In DEAP’s third level, it encodes users’ sequential 
 preferences to capture the time-evolving attendance patterns and interacts with the generated 
 embedding vectors from the first two levels.
With the three-level LSTM architecture, the generated semantic embedding vectors encode 
multi-dimensional preferences (i.e., sequential, contextual, and exclusive preferences). 
Finally, the embeddings are fed into a Multilayer Perceptron (MLP) for 
predicting the event attendance of each user. 

\noindent However, the crime prediction problem in urban sensing remains a challenging 
problem to be solved. Here, an end-to-end model is developed to predict the future crime 
occurrence of each geographical region in a city.

\section{Crime rate inference and detecting crime hotspots}
There exist prior studies on crime rate inference and detecting crime 
hotspots \cite{p7,p2}. 
For example, Wang et al. \cite{p7} aimed to
infer crime rate in a city by utilizing Point-of-Interest 
information. 
POI data provides venue information such as GPS coordinates, category, 
popularity, and reviews. 
These POIs mostly belong to categories such as food, shop, transit, education, etc. 
Using such categorical information of POIs are useful to profile neighborhood 
functions. Such 
neighborhood functions could further help predict crime rate. Their experiments 
showed that 
incorporating POI features significantly improve the crime rate inference. Adding
POI features in addition to demographics features reduced the relative error 
by at least 5\% in 
their experiments. This demonstrated that POI data provides additional 
information about the communities that is not covered by the demographics.

\noindent Yu et al. \cite{p2} developed a boosting-based clustering algorithm
to identify crime hotspots. The main idea of this approach is
to iteratively pick a set of local patterns which give the least classification error
at each boosting round. Each set of local patterns is referred as an ensemble
spatio-temporal pattern and is assigned a score. At the end of boosting, a global 
pattern is constructed from these ensemble patterns. This global pattern is capable 
of predicting crime by scaling the total score of an input, a collection of crime 
indicators, evaluated on each crafted ensemble patterns 

\section{The problem of crime prediction}

This project is closely related to works that study the problem of 
crime prediction \cite{a3, p3} which can be categorized into two groups. 
\begin{enumerate}
    \item \textbf{Statistical methods:}\\[0.2cm]
    Census statistical information was used to discuss crime events, such 
    as demographic information and symbolic racism.\newpage
    
    \item \textbf{Data mining techniques:}\\[0.2cm]
    Gerber et al. \cite{a3} used Twitter data to predict crimes in a city. They pursued 
    three objectives: (1) quantify the crime prediction gains achieved by adding 
    Twitter-derived information to a standard crime prediction approach based on
     kernel density estimation (KDE), (2) identify existing text processing tools
      and associated parameterizations that can be employed effectively in the 
      analysis of tweets for the purpose of crime prediction, and (3) identify
       performance bottlenecks that most affect the Twitter-based crime prediction
        approach. Their results indicated progress toward each objective. They
         have achieved crime prediction performance gains across 19 of the 25 
         different crime types in their study using a novel application of 
         statistical language processing and spatial modeling.\\
    Zhao et al. \cite{p3} addressed the crime prediction problem by considering spatial-temporal
     correlations between regions. They exploited temporal-spatial
      correlations 
     for crime
    prediction with urban data. In essence, their aim was to investigate the following 
    two challenging questions: (i) what temporal-spatial patterns can be observed 
    about crimes with urban data; and (ii) how to model these patterns 
    mathematically for crime prediction. For temporal-spatial patterns, 
    they focused their investigation on (a) intraregion temporal correlation 
    and (b) inter-region spatial correlation. Intra-region temporal correlation 
    tells how crime evolves over time for a region in a city; while inter-region
     spatial correlation suggests the geographical influence among regions in the
      city. They proposed a novel framework TCP, which captures temporal-spatial 
      correlations for crime prediction.
\end{enumerate}

\noindent Most of the above studies forecast the crimes using statistical
or conventional data mining approaches. However, those previous
crime prediction techniques relied on a good amount of high quality
static demographic data or ignored the dynamic temporal dependencies in the 
distributions of crime sequence. In contrast, this work
develops a neural network-based crime prediction model which
jointly models time-evolving dependencies in multi-dimensional
crime data and incorporates both static and dynamic ubiquitous
data (i.e., POI and urban anomaly data) into the framework.


\noindent This work is related to literature that focuses on modeling timestamped
 data. Recently, in light of the significant progress yielded 
 by deep learning techniques on natural language processing and speech, many efforts have been made to apply recurrent neural networks (RNN) and its variants in modeling time series data. 
 For example, Wu et al. \cite{p9} predicted ratings of users for movies with an LSTM 
 architecture by exploring users’ historical behavioral trajectories. 
 They proposed Recurrent Recommender Networks (RRN) that are able to predict
  future behavioral trajectories. This was achieved by endowing both users and
   movies with a LSTM autoregressive model that captures dynamics, in addition
    to a more traditional low-rank factorization. On multiple real-world datasets, 
    their model offers
excellent prediction accuracy and is very compact.

\noindent Laptev et al. \cite{p8} proposed a LSTM-based
architecture for special event forecasting at Uber using heterogeneous 
time-series data. 

\noindent Inspired by the above work, a new neural architecture was developed to 
capture the time-varying patterns in crime sequences and implicit contextual signals
 embedded in relevant ubiquitous data.

