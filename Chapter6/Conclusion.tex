Crime prediction is a challenging and important task, and interpreting 
the time-ordered sequential crime data is a hard and vital
problem for predictive model in urban sensing. This paper explored
the neural network architectures to explicitly model the evolving
dependencies in time-ordered crime sequence and implicit multidimensional 
interactions between regions, categories and time slots.
This paper evaluated the new framework on real-world datasets collected
from NYC. The results showed that this approach achieves better
performance when competing with baselines.
Notwithstanding the interesting problem and promising results,
some directions exist for future work. \emph{First}, the proposed framework is a
 general framework to capture spatial-temporal-categorical dynamics,
It can be applied to a much broader set of urban
data forecasting applications. \emph{Second}, more data sources (\emph{e.g.,} 
social media data) can be explored in addition to the ubiquitous
data used in this work.